%\vspace*{-3ex}
\Header{bugs}{Run WinBUGS from R}
%\vspace*{-3ex}
\begin{Description}\relax
The \code{bugs} function takes data and starting values as input.
It automatically writes a WinBUGS script, calls the model,
and saves the simulations for easy access in R.
\end{Description}
%\vspace*{-3ex}
\begin{Usage}
%\vskip -2ex
\begin{verbatim}
bugs(data, inits, parameters.to.save, model.file = "model.txt",
    n.chains = 3, n.iter = 2000, n.burnin = floor(n.iter/2),
    n.thin = max(1, floor(n.chains * (n.iter - n.burnin)/1000)),
    bin = (n.iter - n.burnin) / n.thin,
    debug = FALSE, DIC = TRUE, digits = 5, codaPkg = FALSE,
    bugs.directory = "c:/Program Files/WinBUGS14/",
    working.directory = NULL)
\end{verbatim}
\end{Usage}
%\vspace*{-3ex}
\begin{Arguments}
\begin{ldescription}
\item[\code{data}] either a named list (names corresponding to variable names in the \code{model.file})
of the data for the WinBUGS model, \emph{or}
a vector or list of the names of the data objects used by the model.
If \code{data = "data.txt"}, it is assumed that data have already been written to the working directory
in a file called \file{data.txt}, e.g. by the function \code{\Link{bugs.data}}.
\item[\code{inits}] a list with \code{n.chains} elements; each element of the list is
itself a list of starting values for the WinBUGS model, \emph{or}
a function creating (possibly random) initial values.
Alternatively, if \code{inits = NULL}, initial values are generated by WinBUGS
\item[\code{parameters.to.save}] character vector of the names of the parameters to save which should be monitored
\item[\code{model.file}] file containing the model written in WinBUGS code.
The extension can be either \file{.bug} or \file{.txt}.
If \file{.bug}, a copy of the file with extension \file{.txt} will be created
in the \code{bugs()} call and removed afterwards.
Note that similarly named \file{.txt} files will be overwritten.
\item[\code{n.chains}] number of Markov chains (default: 3)
\item[\code{n.iter}] number of total iterations per chain (including burn in; default: 2000)
\item[\code{n.burnin}] length of burn in, i.e. number of iterations to discard at the beginning.
Default is \code{n.iter/2}, that is, discarding the first half of the simulations.
\item[\code{n.thin}] thinning rate.  Must be a positive integer.
Set \code{n.thin} > 1 to save memory and computation time if \code{n.iter} is large.
Default is \code{max(1, floor(n.chains * (n.iter-n.burnin) / 1000))}
which will only thin if there are at least 2000 simulations.
\item[\code{bin}] number of iterations between saving of results
(i.e. the coda files are saved after each \code{bin} iterations);
default is to save only at the end.
\item[\code{debug}] if \code{FALSE} (default), WinBUGS is closed automatically
when the script has finished running, otherwise WinBUGS remains open for further investigation
\item[\code{DIC}] logical; if \code{TRUE} (default), compute deviance, pD, and DIC
\item[\code{digits}] number of significant digits used for WinBUGS input, see \code{\Link{formatC}}
\item[\code{codaPkg}] logical; if \code{FALSE} (default) a \code{bugs} object is returned,
if \code{TRUE} file names of WinBUGS output are returned for easy access by the \pkg{coda} package.
\item[\code{bugs.directory}] directory that contains the WinBUGS executable
\item[\code{working.directory}] sets working directory during execution of this function;
WinBUGS' in- and output will be stored in this directory;
if \code{NULL}, the current working directory is chosen.
\end{ldescription}
\end{Arguments}
%\vspace*{-3ex}
\begin{Value}
If \code{codaPkg = TRUE} the returned values are the names
(without file extension) of files written by WinBUGS containing
the Markov Chain Monte Carlo output in the CODA format and corresponding index files.
This is useful for direct access with \code{\Link{read.bugs}} from package \sQuote{coda}.

If \code{codaPkg = FALSE}, the following values are returned:
\begin{ldescription}
\item[\code{n.chains}] see Section \sQuote{Arguments}
\item[\code{n.iter}] see Section \sQuote{Arguments}
\item[\code{n.burnin}] see Section \sQuote{Arguments}
\item[\code{n.thin}] see Section \sQuote{Arguments}
\item[\code{n.keep}] number of iterations kept per chain (equal to \code{(n.iter-n.burnin) / n.thin})
\item[\code{n.sims}] number of posterior simulations (equal to \code{n.chains * n.keep})
\item[\code{sims.array}] 3-way array of simulation output, with dimensions
n.keep, n.chains, and length of combined parameter vector
\item[\code{sims.list}] list of simulated parameters:\\
for each scalar parameter, a vector of length n.sims\\
for each vector parameter, a 2-way array of simulations,\\
for each matrix parameter, a 3-way array of simulations, etc.
\item[\code{sims.matrix}] matrix of simulation output, with \code{n.chains * n.keep} rows and
one column for each element of each saved parameter
(for convenience, the \code{n.keep * n.chains} simulations in
sims.array and sims.list have been randomly permuted)
\item[\code{summary}] summary statistics and convergence information for each
element of each saved parameter.
\item[\code{mean}] a list of the estimated parameter means
\item[\code{sd}] a list of the estimated parameter standard deviations
\item[\code{median}] a list of the estimated parameter medians
\item[\code{root.short}] names of argument \code{parameters.to.save} and \dQuote{deviance}
\item[\code{long.short}] indexes; programming stuff
\item[\code{dimension.short}] dimension of \code{indexes.short}
\item[\code{indexes.short}] indexes of \code{root.short}
\item[\code{last.values}] list of simulations from the most recent iteration; they
can be used as starting points if you wish to run WinBUGS for further iterations
\item[\code{pD}] \code{var(deviance)/2}, an estimate of the effective number of parameters
(the variance is computed as the average of the within-chain variances,
which gives a more reasonable estimate when convergence has not been reached)
\item[\code{DIC}] \code{mean(deviance) + pD}
\end{ldescription}
\end{Value}

